\documentclass[12pt,a4paper,twoside,danish,article]{memoir}

\usepackage[danish]{babel}

\usepackage{memhfixc} % Dynamic updates to memoir class

\setulmarginsandblock{1in}{1in}{*} % 1-inch margins on top and bottom
\setlrmarginsandblock{1in}{1in}{*} % 1-inch margins on left and right
\checkandfixthelayout

\makeatletter % Suppress various auxiliary commands in bib-file
\newcommand\Firstpublished[1]{\expandafter\ignorespaces\@gobble}
\newcommand\Editedby[1]{\expandafter\ignorespaces\@gobble}
\newcommand\biband{\expandafter\ignorespaces\@gobble}
\newcommand\Bookreview{\expandafter\ignorespaces\@gobble}
\newcommand\Moviereview{\expandafter\ignorespaces\@gobble}
\makeatother

\usepackage{natbib} 
\citestyle{chicago} % Chicago Manual of Style citations
\bibliographystyle{dkchicago}

\usepackage{textcomp}
\usepackage[utf8]{inputenc}
\usepackage[T1]{fontenc} % Font encoding

\usepackage{babel} % Babel for Danish constants and formatting

\usepackage{eurosym} % Symbol for \euro

\begin{document}

\frontmatter
\begin{titlingpage}
  \begin{center}
    \mbox{}\vfill
    \vspace{3cm}
    \Large{Årskort: 20105366, xxxxxx,xxxxxxx}\\
    \Large{Simon, Frederik, Anders}\\
    \vspace{10cm}
    Opgave i forbindelse med kurset Eksperimentiel Systemudvikling\\
    Foråret 2013\\
    \vspace{1cm}
    Kursusansvarlig: Susanne Bødker,\\Aarhus Universitet\\
    \today
  \end{center}
  \clearpage
  \tableofcontents*
\end{titlingpage}

\mainmatter

\renewcommand{\baselinestretch}{1.6}\normalsize % Increase line spacing

\renewcommand{\chaptermark}[1]{\markboth{\thechapter.
    #1}{\thechapter. #1}} % Chapter marks
\renewcommand{\bibmark}{\markboth{\bibname}{\bibname}} % Formatting of
                                % default chapter marks
\renewcommand{\tocmark}{\markboth{\contentsname}{\contentsname}}


\chapter{Spørgsmål 1}
Lorem ipsum....


\chapter{Spørgsmål 6b}
Udfordringen omkring \textit{nanobots} i \cite{Kurzweil2004a} har visse utilitaristiske træk. Mennesket har i dette tilfælde lysten til fremskridt. Målet er, at gøre så mange mennesker lykkelige som muligt. I virkeligheden nærmer det sig \textit{act-utilitarism}\footnote{\citep[s.39]{Johnson2001a-2}}, hvor man prøver at forstå konsekvenserne af handlinger uden hensyn til regler og normer. Man følger blot blindt den vej, som bringer mest fremskridt og udvikling. Udfordringen er ikke blot tilskrevet utilitarisme, men \cite{Kurzweil2004a} skaber et karikeret billede af menneskeheden, som bevæger sig upåklædte ind i en farefuld tidsalder. Et utilitaristisk billede. For at undgå at denne udvikling skaber en pludselig negativ effekt, bør det altså være klart at både dyder og pligtfølelse bør tages i brug for at opnå en større kontrol.\\[5pt]
\cite{EtiskRaad2010a} har andre og mere direkte etiske udfordringer omkring omsorgsrelationer med mennesker. Ser man denne udfordring fra et utilitaristisk synspunkt bør man introducerer så mange sociale robotter som muligt i plejesektoren. Derved skaber man flere brugere som opnår tryghed og glæde ved at socialisere med robotterne. Desuden får samfundet flere frie hænder i plejesektoren, som kan bruges til andre opgaver. I sidste ende vil det medføre et løft i den teknologiske udvikling. Hvis flere robotter bliver appliceret i plejesektoren opnår man uden tvivl et større fremskridt i robotteknologi, som også vil kunne anvendes indenfor andre sektorer. Den deontologiske etik appelerer derimod i højere grad til pligtfølelesen. Du bør ikke behandle et andet individ blot som et middel, men som et mål i sig selv.\footnote{se tidligere citat af Immanuel Kant} Konfronterer man denne tankegang med udfordringen inden for velfærdsteknologi bør det være klart at vi har pligt til at være mod andre, som man vil have de skal være mod dig. En pligt som fx består i at give ældre en ret til \textit{ægte} sociale relationer og ikke lade dem simulere dette med en social robot.           



\backmatter
\renewcommand{\baselinestretch}{1}\normalsize % Reset line
                                % spacing
\bibliography{ekspsys.bib}

\end{document}
