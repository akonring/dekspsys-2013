\chapter{Diskussion af løsninger og perspektivering til litteraturen}

\paragraph{Stoltermans Design Complexity}
Vores arbejde med projektet er sket i en sammenhæng, hvor vi har haft et stort metode-katalog at tage stilling til. Flere af disse metoder er blevet anvendt i arbejdet med projektet. Metoderne har drevet designet og som sådant har vores arbejdsproces været meget fokuseret på selve metoderne. Dette har i høj grad karakteriseret vores arbejde, vi har diskuteret proces og iterationer mens vi har arbejdet. Dette har sikret os at vi ikke har taget ubegrunde designbeslutninger og har samtidig konstitueret en væsentlig del af selve vores arbejde.

Vores arbejde med Institut for (X) har ligeledes været karakteriseret ved et meget åbent domæne. Som tidligere beskrevet har organisationsstrukturen været atypisk i forhold til mange andre arbejdspladser. Der eksisterer i høj grad ingen særligt afgrænsede arbejdsopgaver. Dette har ligeledes betydet at domænet, som vi har skullet designe til, har været meget åbent. Det design, som vi er kommet frem til, vil vi argumentere for i høj grad har søgt netop at imødekomme denne problemstilling.

Erik Stolterman introducerer i sin tekst The Nature of Design Practice and Implications
for Interaction Design Research begrebet Design Complexity. Design Complexity defineres som “[...] the complexity a designer experiences when faced with a design situation.”\footnote{\citep{Stolterman}[s.57]}

Begrebet har så at sige to parametre, det ene hænger sammen med det faktum at designeren befinder sig i en virkelighed med uanede muligheder og den anden parameter er selve designprocessernes og -metodernes kompleksitet. Disse to måder at se kompleksiteten på har, jvf. ovenstående, også gjort sig gældende i vores projekt. “An inexperienced designer might suffer from “design paralysis” when confronted with such endless opportunities”\footnote{\citep{Stolterman}[s.57]} skriver Stolterman i forbindelse med dette fænomen og der kan opstå det, der i Schön’s ord kaldes en “messy”\footnote{\citep{Stolterman}[s.57]}\footnote{\citep{Stolterman}[s.60]}, eller rodet, situation.

Konsekvenserne af design kompleksiteten fundet i forbindelse med metodekataloget har præget os på den måde, at vi har brugt en stor del af processen på at diskutere de valgte metoder. Vi har diskuteret hvordan resultaterne fra metoderne skal drive os videre til næste iteration. Vi har diskuteret hvilke metoder der kunne være relevante at bruge. Vi har på den måde ikke bare fået et bedre kendskab til metoderne, men også sørget for at vi i højere grad har anvendt metoder. der har været relevant for projektet på det aktuelle stadie i udviklingsprocessen. 

Eksempelvis var det vigtigt for os i den tidlige del af designprocessen ikke at forgribe os på designet. Det skal forstås på den måde at vi ikke begyndte at diskutere konkrete designforslag før vi havde begrundet det i observationer og user stories. Det har så også gjort at vi har haft det mindre tid til at arbejde på det egentlige design.

Ligeledes har kompleksiteten af selve vores domæne bidraget med en del design kompleksitet. Stoltermans tanker om design kompleksitet gør sig måske især gældende for vores projekt. Design kompleksiteten ser vi som særligt relevant faktor, når man tager organisationens struktur i betragtning. En bedre erfaring og et dybere kendskab til metodekataloget ville ganske sikkert have skabt en anden situation. Det ville måske have været nemmere for os at se hvornår de forskellige metoder ville være relevante at bruge. Realiteten er den, at vi har brugt en væsentlig del af vores arbejde med projektet på at diskutere metodologiske aspekter af processen. Dette er sket på bekostningen af råt output, men har samtidig sikret os at de beslutninger vi har taget er sket i overensstemmelse med fagets metode.

Stoltermans foreslår at man for at håndtere denne designkompleksitet er ved at agere designerly.\footnote{\citep{Stolterman}[s.60]} Det involverer bl.a. at anvende hensigtsmæssige (appropriate)\footnote{\citep{Stolterman}[s.60]} metoder. Da vi valgte ikke at anvende flow modelling, argumenterede vi netop for hvorfor dette ikke var et passende instrument. Så snarere end at bruge metoden for metodens skyld forsøgte vi at tilpasse det til den konkrete (rodede) situation. 

Vores begrænsede erfaring med design, gjorde at vi ikke nåede at lave så meget evaluering på vores endelige prototype som kunne have været hensigtsmæssigt. Evalueringen kunne have været med til yderligere at guide den fysiske udforming af produktet. Vi kan altså også i mindre grad udtale os om, hvorvidt det er lykkedes for os at designe the ultimate particular.\footnote{\citep{Stolterman}[s.59]}

Vi må også konstatere at designkompleksiteten har påvirket vores design process. Et eksempel er hvornår vi kom til fasen for det fysisk design.\footnote{\citep{Benyon}[s.52]} Der gik formentlig længere tid, end hvis vi ikke havde taget metoderne i brug og ligeledes også længere tid, end hvis vi havde været i en organisation, hvor der var en tydelig defineret arbejdsopgave som manglede IT understøttelse, og som brugerne havde tydeligt identificeret på forhånd. Ikke desto mindre begyndte vi først at prototype efter der var et passende belæg for det i form af scenarier.

\paragraph{Prototype}
Vores indledende prototyping indsats fungerede på den måde, at vi sad og diskuterede findings og løsninger, der kunne komme disse findings til livs. På den måde brugte vi også selve designprocessen til at kommunikere design-idéer.\footnote{\citep{Lim}[s.2]} Vi har på den måde diskuteret idéer frem og tilbage, idéer, som har gået i nogle forskellige retninger og derigennem fået afsøgt dele af envisionment-fasen gennem den “designerly” tilgang.\footnote{\citep{Stolterman}[s.61]}

Vi udformede vores første low fidelity prototyper idéer på papir. Det virkede oplagt at det vi arbejde henimod var at udforme en eller anden web-baseret løsning. Derfor virkede denne måde at prototype på også som den mest oplagte. Som Lim et al beskriver den optimale prototype i The Anatomy of Prototypes; “the best prototype is one that, in the simplest and most efficient way, makes the possibilities and limitations of a design idea visible and measurable”\footnote{\citep{Lim}[s.3]} og netop dette synes vi at kunne opnå ved at lave simple paper-based prototypes. 

I samme tekst snakker de også om prototyper som filters henholdsvis manifestations. Når man kigger på filtering dimensionen af vores prototype, kan man argumentere, at ved at lave papir-baserede prototyper kan vi (til en vis grad) vise systemets udseende, som er væsentligt i forhold til at få systemet til at fremstå intuitivt og tiltalende. Der er naturligvis en del forskel i forhold til at opnå det gode design, men i forhold til at bruge disse prototyper i en sparrings situation internt i designgruppen kan vi få  afprøvet idéerne hurtigt. Disse prototyper har en stor ulempe i form af den manglende interaktivitet, hvilket er ærgeligt i forhold til at afprøve det på brugerne, da netop disse observationer vil være med til at afspejle om vores idéer virker intuitive for vores brugergruppe - en parameter der er et af de væsentligste succeskriterier i forhold til vores designs succes.

I forhold til prototypers anden dimension, manifestations, har vi, ved at prototype på denne måde, anvendt papir som materiale, hvilket taget i betragtning af at vi designer med henblik på et webinterface er et udmærket valg. Prototypens resolution var relativt lavt, da det således gjorde os i stand til at diskutere forskellige idéer uden at bruge en uhensigtsmæssig stor indsats på at lave prototyperne. Prototypens scope var primært fokuseret på hvilke elementer, der skulle indgå og hvordan de skulle se ud for at give brugerne det nemmeste overblik.

I vores næste iteration med prototyperne, hvor vi brugte vi et computer-baseret program til at udforme vores prototype fik vi en øget grad af fidelity. Filtering dimensioner for denne prototype viser mange af de samme resultater fra den papirbaserede udgave, dog er udseendet for denne prototype tættere på den endelige design-idé. Fordelen ved det, er at det således er nemmere at visualisere det endelige produkt. Dette er både en fordel i forhold til kommunikationen med brugerne men også i forhold til at bruge prototypen som en slags rettesnor når den visuelle del af den endelige prototype skal udformes. 

Progressionen i forhold til iterationerne med prototyperne afspejler også hvordan vi har forsøgt at bruge prototyperne til en række ting i vores designproces. Med papirprototyperne kunne vi generere idéer og kommunikere designidéer internt og forsøgt samtidig at få sat en konkret manifestation på vores forståelse af brugernes behov. Noget vi også senere fik testet ved at fremvise vores idéer. Det var i mindre grad muligt at lave nogen form for evaluering og test med vores prototyper, igen fordi at nogle af de vigtigste krav til systemet var at det var intuitivt at bruge og fremstår på en flot måde. I forhold til Lim et al har vores indsats med prototyping altså dækket 3 af 4 parametre: “(1) evaluation and testing; (2) the understanding of user experience, needs, and values; (3) idea generation; and (4) communication among designers”.\footnote{\citep{Lim}[s.25]}
