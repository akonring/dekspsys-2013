\chapter{Designprocessen}
I dette afsnit vil vi give en opsummering af projektet og dets designproces.
Først præsenteres den overordnede strategi for vores arbejde med designprocessen, og hvilke årsager og konsekvenser, den har haft.
Herefter gives en opsummering af arbejdsforløbet, som er overvejende redegørende. 
Til sidst gives en mere uddybende behandling af de relevante metoder i vores projekt samt beskrivelse af brugersamarbejdets rolle for designprocessen.

\section{Behandling af designprocessen}
Arbejdet med kursets metoder har helt fra starten af projektet drevet designprocessen frem: Metoderne har bestemt hvad, som skulle ske, og hvornår. Generelt har vi altså ikke foretaget os noget, som metoderne ikke har foreskrevet. Vi har benyttet metoderne som beskrevet i vores metodekatalog i projektets første aflevering, med enkelte undtagelser. En uddybende analyse af arbejdet med de enkelte metoder følger i afsnittet nedenunder.
Beslutningen om at behandle designprocessen så metodisk bunder til dels i en forhåbning om at metoderne ville guide os til et bedre slutresultat og til dels af nødvendighed: I løbet afdesignprocessen har vi flere gange haft gavn af, at kigge frem mod en kommende metode eller gå tilbage til en tidligere metodes findings, hvis vi havde mistet fokus eller var gået i stå. 
Metoderne har på den måde både guidet os i udforskelsen af designrummet, samt hjulpet os til at finde orden og overblik i designprocessen, så vi blev i stand til at behandle den kompleksitet, som ifølge \citep{Stolterman} ligger i arbejdet med et design. En uddybende bearbejdning af dette emne følger i afsnit 5.

\citep{Benyon} opdeler design processen i fire hoveddele: Understanding, envisionment, evaluation og design. Vi har været i løbende berøring med alle fire dele. 
Vores designproces begyndte med understanding. Det var et naturligt udgangspunktet for projektet, at forsøge at opbygge den nødvendige forståelse af domænet. Observationer, interviews og user stories har været centrale elementer i opnåelsen af forståelsen. Først efter denne fase havde vi den nødvendige viden og grundlag for at kunne argumentere for beslutninger og valg af løsninger i forbindelse med envisionment og design. Vores valgte rækkefølge af metoder i metodekataloget afspejler dette.
Vi har forsøgt at behandle evaluering som en central del af processen, ofte i samarbejde med brugerne. Vi har løbende præsenteret vores findings af metoderne for brugerne, ligesom vi har evalueret vores prototyper med dem.

\section{Planlægning og udførsel}
Den første kontakt til organisationen blev påbegyndt i februar. Planlægningen af et egentligt projekt blev påbegyndt i marts, hvor gruppen fik sat deadlines på metoder og andre milepæle i designprocessen. Gruppen forsøgte at opdele processen ud fra de iterationer og evalueringer, som hver metode måtte give anledning til. Et indledende møde\footnote{Bilag E: Dagbog, indlæg 19. marts} med organisationen blev ligeledes afholdt i marts, hvor vi præsenterede projektet og vores metodekatalog og aftalte fremtidige møder.

Henover eksamensperioden stod projektet i ro og vi fik ikke rigtigt afholdt møder i gruppen, som set i retroperspektiv ville have hjulpet os en del: Vi måtte efterfølgende kæmpe mere for at overholde senere deadlines og med at nå bestemte iterationer, inden de skulle behandles i rapporten, og havde derfor problemer med at gå i dybden med visse iterationer, fordi en senere iteration pressede sig på.\footnote{Bilag E: Dagbog, indlæg 17. april}

I april udførte vi en række kontekstuelle interviews og observationer af udvalgte personer fra organisationen. Organisationens mangel på struktur, de udefinerede arbejdsopgaver og  løse arbejdsfordelinger gjorde, at det var svært at opnå en sådan grad af forståelse af organisationen, at vi følte os rustet til at fortsætte med andre metoder. En videre diskussion af dette følger i afsnit 3.

Umiddelbart efter de kontekstuelle interviews begyndte arbejdet med udviklingen af et konceptuelt design. Vi startede med at opsummere vores findings fra de kontekstuelle interviews for at forberede os på det forestående arbejde med work modelling\footnote{\citep{Benyon}[s. 277]}.
I overensstemmelse med vores metodekatalog, begyndte begyndte vi herefter udformningen af en række flow modeller, som skulle tydeliggøre strukturen og arbejdsflowet i organisationen og hjælpe os med at lokalisere eventuelle breakdowns. 
Vi fandt dog hurtigt, at organisationens vante praksis, med diffuse arbejdsopgaver og -fordelinger, gjorde det svært, at præcisere hvilket arbejde, som helt konkret blev udført. Af den grund havde vi en del problemer med at få udformet en flow model\footnote{Bilag E: Dagbog, indlæg 1. maj}, som vi følte på passende og tilstrækkelig vis indfangede en væsentlig del af det pågående arbejde på instituttet. Efter flere forsøg med ikke-fyldestgørende resultater valgte vi derfor at udelade flow modellerne; en uddybning og diskussion af denne beslutning følger i afsnittet nedenfor om flow modeller.

I kølvandet på vores beslutning om at genoverveje flow modellerne, blev vi enige om at gå over til en scenarie-baseret arbejdsproces og lægge work-modelling på hylden.\footnote{Bilag E: Dagbog, 2. maj} Vi fandt, at vi var bedre udrustet til en sådan tilgang til projektet: Vores forståelse af organisation var i høj grad baseret på user stories og ikke direkte observationer af den konkrete arbejdsgang. Med udgangspunkt i disse user stories, skabte vi en række konceptuelle scenarier: De skulle abstrahere unødvendige detaljer ud og finde ind til kernen af historierne, og på den måde gøre det tydeligere, hvad brugerne ville fortælle. 

Igennem vores refleksioner om flow modellerne og vores manglende forståelse for strukturen i organisationen, blev det tydeligere for os, at vi ikke skulle søge at finde et design, som kunne passe ind i en specifik arbejdsopgave, men rettere måtte virke på tværs af projekter og arbejdsopgaver og uanset hvilke arbejdsfordelinger, som måtte finde sted. Det blev klart, at der ikke var én opgave, som skulle behandles, men at den reelle aktivitet i organisationen var “arbejdet med projekter” og at designet måtte afspejle dette. Et begyndende konceptuelt design blev dermed tydeligere.

Det konceptuelle design blev videre tydeliggjort i arbejdet med konkrete scenarier og use cases, som fulgte bagefter arbejdet med de konceptuelle scenarier. Disse blev skabt med en klarere forestilling om et design: Vi har fremhævet bestemte features og fravalgt andre, ligesom vi ikke forsøger at løse alle problemstillinger eller breakdowns i organisationens arbejdsgang. I stedet behandler vi først og fremmest problemstillinger i forbindelse med “arbejdet med et projekt”: Få struktur på filer, få en veldefineret kalender, hold styr på medlemmer og intern kommunikation, og eventuelt få opbygget en ordentlig profil og præsentation af projektet til offentligheden. Vores scenarier har præsenteret forslag til løsninger på disse problemer gennem brugs-eksempler.

De samme problemstillinger har været i fokus i arbejdet med prototyping. Først har vi skitseret vores løsningsforslag på papir og senere elektronisk, gennem et software-baseret prototyping program\footnote{Bilag C: Lo-fi prototype}. Slutteligt har vi udformet den indledende del af den endelige prototype\footnote{Bilag A: Hi-fi prototype}. Vi har præsenteret prototyperne for en repræsentant for organisationen, som efterfølgende kom med konstruktivt kritik og forslag til forbedringer. Vi har herefter forsøgt at implementere en række udvalgte features fra vores lo-fi prototyper. Vi har taget udgangspunkt i \citep{Lim} og dets forestilling om prototyper som filtre. Det har derfor været vigtigt for os kun at prototype de elementer og dele af vores løsningsforslag, som måtte være interessante at bearbejde dybere. En videre diskussion af dette emne findes i afsnit 5.

\section{Metoder}
Her følger en dybere behandling af de metoder, som gruppen har berørt i løbet af designprocessen. Vi vil her forsøge at diskutere brugen og udfaldet af enkelte af metoderne samt hvad, der gik godt og hvad, der gik skidt.

\subsection{PACT-analyse}
PACT-analysen har været en væsentlig metode i vores værktøjskasse helt fra begyndelsen: Som en del af de obligatoriske opgaver og som et værktøj til analyse og udforskning af domænet for designet.
Vi er igen og igen vendt tilbage til PACT-analysen i forbindelse med problemer eller udfordringer i processen: 
\begin{description}
\item[$\bullet$] Hvilke metoder er relevante at benytte?
\item[$\bullet$] Hvem skal vi interviewe? Hvad skal vi spørge dem om?
\item[$\bullet$] Hvem er det nu lige, at vores brugergruppe er? Hvad laver de?
\item[$\bullet$] Hvilken opgave forsøger vi at løse? Rammer dette design overhovedet den?
\end{description}
Som en del af de obligatoriske opgaver har vi flere gange fået mulighed for at genoverveje vores forståelse af Instituttet og revidere vores PACT analyse, ligesom vi løbende har præsenteret vores analyse for brugerne på Instituttet. På den måde har PACT analysen også hjulpet med at holde fokus i processen.

\subsection{Contextual Inquiry}
Vores forståelse af organisationen, og dermed hele grundlaget for udformelsen af et design, er blevet opbygget gennem observationer, indsamling af user stories og semi-strukturerede interviews, som er foretaget på Instituttet i arbejdstiden. 
Således danner de kontekstuelle interviews (i stærkt sammenspil med PACT analysen) fundamentet i vores designproces, på hvilken alle de senere metoder bygger. Ifølge \citep{Benyon} er det ud fra denne forståelse af brugerne og brugernes arbejdsgang, at udviklingen af kravspecifikationer til et design kan finde sted.
Vi besøgte Instituttet fire gange med det formål at udvikle vores forståelse af organisationen. Her har vi foretaget semistrukturerede interviews af tre udvalgte nøglepersoner, observeret arbejdsgangen på Instituttet, siddet med til møder og snakket med tilfældige brugere, som har vist interesse i vores projekt. En stor del af det indsamlede data er user stories.
Vi følte flere gange en utilstrækkelighed i vores forståelse af organisationen, ligesom vi havde svært ved at formulere præcist, hvad der fandt sted i organisationen. Af tidsmæssige årsager måtte vi dog til sidst erkende, at vi ikke kunne tilbringe mere tid med CI og at vi derfor måtte stille os tilfreds med den viden, som vi havde tilegnet os. 

\subsection{Computer-supported Cooperative Work Analyse}
For at sætte ekstra fokus på CSCW på Instituttet og hvilke teknologier, som anvendes til samarbejde om projekter har vi udarbejdet en tabel, som anvender \citep{Penichet} til at klassificere teknologierne. \citep{Penichet} udmærker sig ved både at have den gængse space/time-kategorisering som findes i DeSanctis og Gallupe\footnote{\citep{Benyon}[kap. 18]}, samt karakteristikkerne information sharing, communication og coordination. Desuden giver denne kategorisering mulighed for at anerkende en teknologi, for eksempel Google Docs, som værende en platform for kollaboration, men også som kommunikationsmedie. 
Vi anvendte dette framework til at gå yderligere i dybden med teknologierne fra PACT analysen. Klassificeringen af teknologierne dannede et bedre overblik. Vi fandt ud af hvilke værktøjer som brugerne allerede havde kendskab til og hvilke kvaliteter de havde. Ligeledes var det tydeligt at bemærke, hvilke kommunikationsværktøjer som overlappede i den forstand at de havde de samme teknologiske kvaliteter. I den forbindelse var det vigtigt at oplyse om disse overlap og forsøge at gøre brugeren opmærksom på forskellige værktøjer og egenskaberne hos disse. 

\subsection{Flow modelling}
Som tidligere nævnt var arbejdet med flow modellerne svært, og vi mener at grunden hertil skal findes i det utraditionelle arbejde, som udføres i Instituttet. Hvordan beskriver man arbejdsflowet i en organisation, hvor ingen medarbejdere har faste mødetider og hvor man påtager sig en arbejdsopgave forskellig fra dag til dag og projekt til projekt? Hermed har vi ikke sagt at det er en umulig opgave, idet der sagtens kunne eksisterer en eller anden arbejdsfordeling, som det er muligt at modellere, men blot at det er svært at udlede en sådan model ud fra enkelte observationer og user stories. Vi bemærker at brugerne selv har svært ved at beskrive præcist hvad der foregår på stedet og hvem som gør hvad.
Vi har alligevel udviklet et par flow modeller ud fra de data, som blev indsamlet i de foregående faser, men vi fandt dem upræcise og ude af stand til tilstrækkeligt at indfange det virkelige arbejde, som finder sted på Instituttet. 
Vores flow modeller tog i første omgang udgangspunkt i det generelle arbejdsflow. Det viste sig at det var svært at identificere nogle generelle træk. Derefter tog modellerne udgangspunkt i en konkret arbejdssituation, som vi havde observeret. Selv med en konkret opgave for øje kunne vi ikke præcist nok specificere, hvilke individer og roller som var tilstede, eller hvorledes de kommunikerede.
Vi overvejede at vi gennem evaluering med organisationen muligvis kunne afklare flere af disse forvirringer, men vi besluttede at lade være. Vi mente ikke at netop denne arbejdsopgave ville fortælle os noget nyt om organisationen, og at vores problemer med metoden i stedet måtte bunde i at flow modellerne ikke var en passende værktøj til at modellere strukturen i netop denne organisation.
Efterfølgende er det blevet tydeligere hvorledes en arbejdsfordeling måtte udforme sig i forbindelse med et projekt; der er bestemte roller, som ofte går igen imellem projekterne og som man muligvis kunne modellere. Men netop i kraft af, at det generelle arbejde i organisationen er så uklart, ville det dog være nødvendigt enten at formulere modellerne meget specifikt eller meget generelt: Man kunne udvikle en model efter et enkelt, konkret projekt med dets specifikke arbejdsopgaver og -roller, hvorefter man intet generelt ville kunne uddrage af modellen, eller man kunne udvikle modellen efter det “generelle projekt”, hvorefter man igen ville få svært ved at identificere nok gennemgående roller og arbejdsopgaver til at konstruere en hel model.
Et godt kompromis ville muligvis være at udfærdige en hel række af specifikke modeller og så holde dem op mod hinanden i håbet om at kunne identificere generelle roller og breakdowns, på samme måde som Benyon påpeger at det er nødvendigt at udforme og sammenholde flow modeller ud fra forskellige individers synspunkter. Havde vi haft mere tid i løbet af denne iteration, ville vi have haft mulighed for at gøre dette.

\subsection{Scenarier og use cases}
På baggrund af de user stories, som blev indsamlet i forbindelse med CI, har vi arbejdet med udviklingen af en række scenarier. Vi fulgte her Benyons model for et scenariebaseret design: Igennem sammenligning og abstraktion af user stories har vi først udformet en række konceptuelle scenarier, som herefter har virket som grundlag for udformningen af konkrete scenarier.
Med inspiration i Benyons eksempel på et konceptuelt scenarie, har vi skabt vores konceptuelle scenarier centreret omkring det fremtidige design og brugerens interaktion med dette. Vores konceptuelle scenarier er på den måde ikke blot en abstraktion af vores user stories, men også et forsøg på generere design idéer og forstå kravene til systemet. De konkrete scenarier er en forlængelse af de konceptuelle scenarier, og sætter brugeren og brugen af systemet ind i en dybere kontekst, så det er nemmere at evaluere konkrete idéer. 
Senere har vores vejleder fastslået Benyons eksempel som upræcist og forklaret at de konceptuelle scenarier ikke bør tage udgangspunkt i designet. 

\subsection{Prototyping}
Prototyping har været en del af metodekataloget fra begyndelsen af projektet. Strategien for processen har fra begyndelsen været brugerorienteret og prototyping har således været et vitalt værktøj, da netop denne metode har egenskaben at at kunne involvere brugeren og andre stakeholders i designprocessen og evalueringen af designet. Se også afsnittet om prototyper som filtre i sidste kapitel. Vi præsenterer resultaterne af arbejdet med prototyperne i næste kapitel sammen med den endelige prototype.

\subsubsection{Lo-fi prototyper}
Gruppens første arbejde med at prototype blev foretaget i små, hurtige iterationer: Gruppen udarbejdede papir-prototyper, som var nemme at producere, med henblik på at kunne afprøve mange forskellige idéer til både det konceptuelle og det fysiske design. Specielt førstenævnte stod dog i fokus, idet meget af den tidlige prototyping indsats omhandlede valg og fravalg af funktionalitet.

Gruppen præsenterede papir-prototyperne for Valentin med henblik på at evaluere dem. Det var vigtigt for gruppen at præsentere prototyperne i en kontekst: Vi anvendte derfor flere brugsscenarier til at guide Valentin mod de kvaliteter som vores prototyper forsøgte at fremhæve. Det var tydeligt at den fysiske fremstilling af designet gav anledning til mange nye designideer (jvf. prototyping som idea generation), og vi gennemførte endnu en iteration af prototyping med Valentin, i det vi i samarbejde med ham hurtigt skitserede nogle nye idéer på papir.

Som en senere del af Lo-fi prototyping overførte vi de skitserede papir-prototyper til det digitale medie igennem et software program. Dette blev gjort med henblik på at tydeliggøre vores design idéer, således
at funktionaliteten af det konceptuelle design stod klarere frem. Senere hjalp disse prototyper også med udformningen af hi-fi prototypen, som lå i naturlig forlængelse af disse.

\subsubsection{Hi-fi prototype}
Denne prototype blev en umiddelbar forlængelse af vores lo-fi prototype efter et antal af iterationer og evaluering med brugere fra Instituttet. Af tidsmæssige årsager nåede vi ikke at afslutte
denne iteration: Vi har endnu ikke evaluereret prototypen med vores brugergruppe. Vi bemærker dog at dette ville være det næste skridt i vores designproces, hvis vi havde haft mere tid.
Det er kun få iterationer, som har behandlet den egentlige hi-fi prototype. Se bilag A. Vi nåede dog at stifte bekendtskab med denne del af processen og vi blev opmærksomme på, hvor anvendeligt værktøjer som agil udvikling og SCRUM er i disse iterationer.

For gruppen har denne del af processen været mere håndgribelig end de tidligere. De tidlige iterationer havde til formål at få en forståelse af organisationen. Dette gav en kompleks sammenhæng mellem iteration og metoder, som ofte var svær at beskrive og få overblik over. Prototype-fasen har gjort at iterationerne er blevet mere veldefinerede og gruppen har haft mulighed for at dokumentere, evaluere og redesigne. Denne forholdsvis faste struktur af iteration, samt det at gå fra konceptuelt objekt til fysisk objekt gjorde designprocessen mere overskuelig. 

\section{Samarbejdet med brugerne}
Brugersamarbejdet har spillet en central rolle i vores projekt. Vi har brugt en del tid sammen med brugerne indenfor rammerne af Instituttet. Vi har gennem dem forsøgt os at danne et overblik over det arbejde, der foregår på stedet og om hvilket vi har forsøgt at lokalisere breakdowns, som vi kunne designe løsninger til.
Vores samarbejde har i særlig grad omfattet: Mads-Peter Laursen, som er daglig leder af Institut for (X) samt Yvonne Buer og Simon Valentin, som står for ledelsen af en af afdelingerne i Instituttet kendt som “B//huset”. Desuden har vi snakket med en række tilfældige brugere, som vi har mødt på Instituttet og som har vist interesse i vores projekt.

Vi observerede Laursen i forbindelse med en arbejdsdag på instituttet. Det var en typisk dag, hvor han foretog sig atypiske opgaver, i den forstand at hver dag er uforudsigelig og består af uventede opgaver. Laursen havde desværre ikke tid til at give os et egentligt interview, men vi fik et indblik i hans arbejdsmetoder samt dannet os en forestilling om en central person på organisationen.

Den oprindelige kontakt til organisationen blev formidlet gennem Valentin og Buer. Vi har tilbragt tid sammen med Valentin mens han har arbejdet med planlægningen af et arrangement hvor han søgte midler. Vi har desuden udført et detaljeret kontekstuelt interview med Valentin, som vi også filmede og klippede sammen til en kort video, der viste nogle af hovedpointerne fra det interview. 
Valentin har desuden forsynet os med feedback på vores designidéer og prototyper, ligesom vi løbende har præsenteret resultaterne af vores analyser af organisationen og dets arbejdsstruktur for ham. Både Valentin og Buer viste stor interesse i at få del i vores findings: De gav udtryk for at de ville have glæde af at blive opmærksom på breakdowns i deres arbejdsgang, ligesom de generelt kæmpede med at få et overblik over aktiviteterne i organisationen. De nævnte endvidere at det ofte var et problem at forklare omverdenen hvad der foregik i organisationen, hvilket er problematisk for en organisation, hvis indtægter hovedsageligt kommer fra fundraising og kommunal støtte.

Brugersamarbejdet har givet os adgang til en dybere forståelse af nogle af de aktiviteter, som foregår på instituttet. Der er stadigvæk en del af brugerne, som vi ikke har kommet i kontakt med og dette forhold er naturligvis ikke uvæsentligt. Dette er en begrænsende faktor for almen gyldigheden af vores brugerundersøgelse, men ligger samtidig tæt op af observationen, omkring organisationens usædvanlige struktur. Vi bemærker endnu engang at organisations interne arbejdsgange og -strukturer fremstår vage og udefinerbare og at vi i vores designproces har beskræftiget os meget med at danne overblik over organisationen og dets arbejde. På den måde kan man argumentere for at organisationens struktur (eller mangel på samme) har bidraget til at forstørre roderiet og kompleksiten, som er en naturlig del af design processen. En uddybende diskussion af dette samt en perspektivering til Stolterman følger i sidste kapitel.
