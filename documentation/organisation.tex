\chapter{Organisationen}
Ifølge \citep{Benyon} er det væsentligt i brugerorienteret design at tage udgangspunkt i de mennesker som bevæger sig (læs: har aktiviteter) i domænet. Her menes både den direkte bruger af det færdige system og de mere perifere stakeholders. PACT frameworket\footnote{\citep{Benyon}[s.26]} er et værktøj, som er blevet anvendt gentagne gange i designprocessen. 
Dette afsnit er derfor et resultat af flere iterationer, gennem hvilke vi har reevalueret PACT og løbende har opbygget en forståelse af domænet. Det tilstræbes at læseren ligeledes får et godt billede af organisationen. Dette gøres tydeligere ved at præsentere en user story, samt belyse emnerne; kommunikation og samarbejde på Instituttet.     

\paragraph{Aktiviteter}
Instituttet er en paraplyorganisation for forskellige projekter. Projekterne omhandler ofte kulturudvikling, opsætning af musikarrangementer eller kunstudstillinger, design, arkitektur, startups, diverse konstruktioner i træ og metal og andre kreative udfoldelser. Udtænkningen og eksekveringen af projekterne danner i høj grad grundlag for de aktiviteter, som finder sted på Instituttet: tegning og design, planlægning og fundraising, byggeri og event-afholdelse. Projekternes størrelse og kompleksitet varierer fra de helt små (bygning af højbed på taget) til de meget store (musikfestival for en halv million). 

\paragraph{People}
Arbejdet i projekterne er i næsten alle tilfælde drevet af ulønnede frivillige, som ønsker at være en del af et projekt og som dermed selv har påtaget sig et ansvarsområde specifikt til opgaven. De involverede er typisk 18-30 år og uden særlig uddannelse eller erhvervsmæssig erfaring indenfor domænet. Derimod har de en stor portion energi og gå-på-mod. Mange personer er med i flere, samtidige projekter og man kan sagtens være leder eller tovholder i ét projekt og kun bidrage sporadisk i et andet. På trods af at enkelte personer har en fast rolle (“daglig leder”, etc.), er Instituttet således en utraditionel virksomhed uden fast ledelse, hierarki eller en veldefineret uddelegering af ansvarsområder.

\paragraph{Context}
Instituttet har efter aftale med kommunen lejet lokaler i en række gamle bygninger på Godsbanen i Århus. Her er værksteder, kontorer, scener, udstillingslokaler m.m, som danner en platform for arbejdet med projekterne.
De engagerede parter ved, at Instituttet kun er et midlertidigt foretagende: Den
24. december 2014 ophører aftalen med kommunen og instituttet bliver smidt ud. Datoen kendes blandt brugerne som “bulldozer day” og er ikke kun en begrænsende faktor; den motiverer til en carpe diem-filosofi i arbejdet med projekterne.
Der er ofte få eller ingen økonomiske midler til rådighed til projekterne, udover dem, som søges ved fundraising.

\paragraph{Technologies}
Mange af projekterne involverer typisk en gruppe af mennesker og kræver ofte en del koordinering og kommunikation. Ofte tildeles hvert projekt en gruppe på det sociale medie Facebook; nogle projekter har endda to (ekstern vs. intern kommunikation). Flere af brugerne på instituttet har dog ikke en Facebook-bruger og nogle føler at de går på kompromis med dataintegriteten, når eksterne værktøjer anvendes.
E-mail bruges også flittigt, selv om det også kan resultere i komplikationer (“svar til alle og ikke kun til mig”, ”jeg har ikke Daniels mail”, osv.)
Google Drive bruges som regel til håndtering og deling af vigtige filer, såsom projektbeskrivelser, budgetter og billeder. De brugere, som vi mødte, var generelt glade for Google Drive, men bemærkede at det ofte blev rodet og ustruktureret, specielt når man arbejder på flere projekter samtidig.

\paragraph{Computer-supported Cooperative Work}
Kommunikation og samarbejde er væsentlige emner på Instituttet og i forståelsen af organisationen. Computer-supported Cooperative Work\footnote{\citep{Benyon}[s. 440]} (CSCW) og kommunikation kombinerer mange aspekter af PACT. For eksempel er kommunikationsmidler technologies, mens de kommunikerende eller samarbejdende parter er people. Teknologierne på Instituttet vil senere i opgaven blive belyst. Dette med henblik på at kategorisere teknologierne og skabe et overblik over, hvilke egenskaber som Instituttet sætter pris på ved teknologierne.
 
\paragraph{User story}
Følgende user story er observeret af gruppen og er et eksempel på hvorledes der kan opstå breakdowns i den interne kommunikation i et projekt:
I forbindelse med planlægningen af Aarhus Lydfestival indkalder tovholderen for projektet, Valentin\footnote{Simon Valentin, Institut for X}, til fællesmøde. Mødet finder sted på Godsbanen en mandag eftermiddag. Indkaldelsen sker via e-mail, men Valentin har ikke alles e-mail adresser; han beder om at mailen bliver videresendt til dem som mangler. Til mødet bemærker Daniel\footnote{Daniel Nielsen, Institut for X} at han ikke har modtaget Daniels mail.
Det er på forhånd umuligt alene ud fra e-mail kommunikationen at vurdere, hvilke personer som dukker op til mødet, og selv om Simon Valentin på dagen sender endnu en e-mail samt en besked via Facebook (“husk møde i dag!”), mangler der alligevel en række personer til mødet.
Til mødet bliver der gennemgået nogle aktuelle dokumenter, som er delt via Google Drive. Ikke alle kan finde dem, og ikke alle har adgang til dem.
Planlægningen af Lydfestivalen er endvidere delt op blandt en række mindre grupper, som hver især har lavet deres egne dokumenter (budgetter, spilleplan, osv). Det skaber en del forvirring og en masse redundant information. Som tovholder for projektet, er det Valentins opgave at flette dokumenterne sammen i sidste ende, hvilket er et stort og kedeligt stykke arbejde.
I slutningen af mødet kommenterer én, at meget af mødets indhold kunne være kommunikeret på andre måder og at mødet derfor har været unødvendigt.
