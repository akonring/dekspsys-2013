\chapter{Introduktion}\label{marker}
Denne opgave har som formål at beskrive, analysere og perspektivere en designproces, som har fundet sted i forbindelse med kurset Eksperimentel Systemudvikling på Aarhus Universitet. 
Vi har til dette projekt oprettet et samarbejde med organisationen, Institut for (X) (Instituttet), ved Godsbanen i midten af Aarhus.

Gennem processen har vi løbende dannet en forståelse af organisationens virke og har gennem udvalgte analysemetoder, udviklet koncept og prototypen på et system til netop dette domæne.
Ved belysningen af denne proces har vi valgt at lægge fokus på, hvilke udfordringer designprocessen har givet anledning til, på hvilken måde disse udfordringer er blevet håndteret, samt reflekteret over hvordan man alternativt kunne have navigeret i domænet og styret designprocessen i disse situationer.

Designet er formet gennem iterationer. Denne opgave beskæftiger sig hovedsageligt med de tidlige dele af processen. Opgaven vil evaluere metoderne, strategien og diskutere udfordringer.

Desuden vil designprocessen, i sin eksperimentelle natur, være brugerorienteret og visse iterationer vil beskrive, hvorledes vi i et vist omfang har måttet ændre strategi og styre designprocessen i en anden retning for at håndgribeliggøre domænet.

