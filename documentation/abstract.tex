\chapter{Abstract}
This report describes and discusses a design process conducted by the authors in a period from March to the end of May 2013. The project was carried out in collaboration with Institut for (X), a cultural institution, where entrepreneurs from different fields come together and carry out different projects. 

The purpose of the design process was firstly to identify a part of the internal workflow of the organisation which would somehow benefit from the support of a novel IT-solution, and secondly to invoke and use a series of methods to clarify the demands and specifications of said solution, and thirdly to approach a final design through iterations of prototyping and user evaluation. 

The resulting design was therefore based on a range of design methods and user involvement and feedback. Methods employed includes: PACT-analysis, user stories and scenarios, prototyping, contextual interviews. Guidelines for structuring a design process dictated the way the group worked with the design throughout. 

The report proposes a final conceptual design. The conceptual design details the structure of a project management and collaboration platform included with a method for presenting the resulting project to the public. Based on these ideas work was carried out on an implementation of this system. The resulting prototype mainly illustrates the look and feel of the final productm but does not have the same degree of interactivity as the system should have to be able to do any conclusive evaluation of it.
